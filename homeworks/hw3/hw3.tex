\documentclass[10pt]{article}

\usepackage{graphicx}
\usepackage{amsmath,amsfonts,amssymb}

\usepackage{hyperref}  % for urls and hyperlinks


\setlength{\textwidth}{6.2in}
\setlength{\oddsidemargin}{0.3in}
\setlength{\evensidemargin}{0in}
\setlength{\textheight}{8.9in}
\setlength{\voffset}{-1in}
\setlength{\headsep}{26pt}
\setlength{\parindent}{0pt}
\setlength{\parskip}{5pt}




% a few handy macros

\newcommand\matlab{{\sc matlab}}
\newcommand{\goto}{\rightarrow}
\newcommand{\bigo}{{\mathcal O}}
\newcommand{\half}{\frac{1}{2}}
%\newcommand\implies{\quad\Longrightarrow\quad}
\newcommand\reals{{{\rm l} \kern -.15em {\rm R} }}
\newcommand\complex{{\raisebox{.043ex}{\rule{0.07em}{1.56ex}} \hskip -.35em {\rm C}}}


% macros for matrices/vectors:

% matrix environment for vectors or matrices where elements are centered
\newenvironment{mat}{\left[\begin{array}{ccccccccccccccc}}{\end{array}\right]}
\newcommand\bcm{\begin{mat}}
\newcommand\ecm{\end{mat}}

% matrix environment for vectors or matrices where elements are right justifvied
\newenvironment{rmat}{\left[\begin{array}{rrrrrrrrrrrrr}}{\end{array}\right]}
\newcommand\brm{\begin{rmat}}
\newcommand\erm{\end{rmat}}

% for left brace and a set of choices
\newenvironment{choices}{\left\{ \begin{array}{ll}}{\end{array}\right.}
\newcommand\when{&\text{if~}}
\newcommand\otherwise{&\text{otherwise}}
% sample usage:
%  \delta_{ij} = \begin{choices} 1 \when i=j, \\ 0 \otherwise \end{choices}


% for labeling and referencing equations:
\newcommand{\eql}{\begin{equation}\label}
\newcommand{\eqn}[1]{(\ref{#1})}
% can then do
%  \eql{eqnlabel}
%  ...
%  \end{equation}
% and refer to it as equation \eqn{eqnlabel}.  


% some useful macros for finite difference methods:
\newcommand\unp{U^{n+1}}
\newcommand\unm{U^{n-1}}

% for chemical reactions:
\newcommand{\react}[1]{\stackrel{K_{#1}}{\rightarrow}}
\newcommand{\reactb}[2]{\stackrel{K_{#1}}{~\stackrel{\rightleftharpoons}
   {\scriptstyle K_{#2}}}~}

% Parts:

% set enumerate to give parts a, b, c, ...  rather than numbers 1, 2, 3...
\renewcommand{\theenumi}{\alph{enumi}}
\renewcommand{\labelenumi}{(\theenumi)}

% set second level enumerate to give parts i, ii, iii, iv, etc.
\renewcommand{\theenumii}{\roman{enumii}}
\renewcommand{\labelenumii}{(\theenumii)}

  % input some useful macros

\begin{document}

% header:
\hfill\vbox{\hbox{AMath 585}
\hbox{Homework \#3}\hbox{Due Thursday, February 6, 2020}}

\vskip 5pt

Homework is due to Canvas by 11:00pm PDT on the due date.

To submit, see
\url{https://canvas.uw.edu/courses/1352870/assignments/5237966}


%--------------------------------------------------------------------------
\vskip 1cm
\hrule
{\bf Problem 1.}
Consider the nonlinear boundary value problem 
\[
\sin(u''(x)) = u(x)\exp(u'(x)) + f(x)
\]
for $0\leq x \leq 1$, 
with Dirichlet boundary conditions $u(0) = \alpha, ~u(1) = \beta$.

(a) Discretize using the standard second-order centered approximations for
$u'(x_i)$ and $u''(x_i)$, giving a nonlinear system of equation $G(U)=0$
where $G:\reals^m \rightarrow \reals^m$ and $U$ is the vector of interior
unknowns.  What is the $i$'th component  $G_i(U)$?

(b) What is the $(i,j)$ element of the Jacobian matix $G'(U)$
needed to implement Newton's method for this system? 

(c) Is the Jacobian matrix symmetric in general? 


% uncomment the next two lines if you want to insert solution...
%\vskip 1cm
%{\bf Solution:}

% insert your solution here!

%--------------------------------------------------------------------------
\vskip 1cm
\hrule
{\bf Problem 2.}

Consider the linearized pendulum problem of Section 2.16, 
which we will rewrite with $x$ in place of $t$ and now calling the angle
$u(x)$ for consistency with our other BVPs:
\[
u''(x) = -\gamma^2 u(x)
\]
on the interval $0 \leq x \leq 1$, with $\gamma^2 = g/L$.  

(a) Confirm that the general solution to this ODE is 
$u(x) = c_1 \sin(\gamma x) + c_2 \cos(\gamma x)$.

(b) If we impose Dirichlet boundary conditions $u(0)=\alpha$ and $u(1) =
\beta$, show that this leads to a $2\times 2$ linear system of equations to
determine the coefficients $c_1$ and $c_2$. Solve this system for the case
$\gamma = \pi/2$ to find $c_1$ and $c_2$ in terms of $\alpha$ and
$\beta$. 

(c) Show that if $\gamma = \pi$ then the system for $c_1, c_2$ is
singular.  For what choices of $\alpha,\beta$ does the system have a
solution? In such cases it has infinitely many solutions, what are they?

(d) Recall that $-\pi^2$ is an eigenvalue of the operator
$\partial_x^2$ on the interval $[0,1]$ with homogeneous boundary conditions
$u(0)=u(1)=0$.  The linear pendulum equation can be written as ${\cal L}u = 0$,
where $L = \partial_x^2 + \gamma^2$.
What are the eigenvalues and eigenfunctions of this operator and how do
these relate to your answer to part (c)?

(e) Suppose we discretize this BVP with the usual second-order centered 
approximation, giving a tridiagonal matrix $A$ for the system for interior
unknowns $[U_1,~\ldots,~U_m]$.  What are the eigenvalues and eigenvectors of
this matrix?  Hint: if $T$ is the tridiagonal matrix of (2.10) then 
$A = T + \gamma^2 I$.

(f) Note that if $\gamma = \pi$ then this matrix is not singular (all the
eigenvalues are nonzero) for $h>0$, 
so the discrete system has a unique solution for any $h>0$.
However, we expect trouble as $h \goto 0$, since we are approximating a BVP
that does not have a unique solution for this choice of $\gamma$.  Show that
in the 2-norm the method is {\em not stable} in the sense of Definition 2.1 
in this case.  How rapidly does $\|A^{-1}\|_2$ grows as $h\goto 0$.
(E.g. like $1/h$?  or $1/h^2$?)

(g) On the other hand, for any $\gamma$ that is not an
integer multiple of $\pi$, show that this method is stable in the 2-norm.

% uncomment the next two lines if you want to insert solution...
%\vskip 1cm
%{\bf Solution:}

% insert your solution here!


%--------------------------------------------------------------------------
\vskip 1cm
\hrule
{\bf Problem 3.}

If $A \in \reals^{m \times m}$ and $u\in\reals^m$ is any nonzero vector, then 
the scalar value $Q(u) = u^TAu / u^Tu$ is called the Rayleigh quotient.

(a) Show that if $v$ is an eigenvector of $A$ then $Q(v) = \lambda$, the
corresponding eigenvalue.

(b) If $A = A^T$ is a symmetric matrix then the eigenvalues must be real. 
The  matrix is called {\em
symmetric positive definite} if the eigenvalues of $A$ are all positive,
or {\em symmetric negative definite} if they are all negative. 


Show that if $A$ is symmetric positive or negative definite then
\begin{equation}\label{Qbound}
\min_p  \lambda_p \leq Q(u) \leq \max_p \lambda_p.
\end{equation}
Hint: Recall that if $A$ is symmetric then it is diagonalizable and the
eigenvectors are mutually orthogonal, so we can write $A = V\Lambda V^T$
where $V$ is the matrix of normalized eigenvectors (each column $v_j$ has
2-norm equal to 1). So $V$ is an ``orthogonal matrix'' with  
$V^{-1} = V^T$. Hence any vector $u\in\reals^m$ can be written as $u=Vy$ for
$y = V^Tu$.  (See also Appendix C.)

Conversely, show that if $A$ is symmetric and if
there are constants $C_1,C_2$ such that
$C_1 \leq Q(u) \leq C_2$ for all nonzero vectors $u$, then the eigenvalues 
of $A$ all satisfty $C_1 \leq \lambda \leq C_2$.

Investigating $Q(u)$ can sometimes help us to show that the eigenvalues of
$A$ are bounded away from 0, which can be useful in proving stability
of a method.

The next problem illustrates one such case.


% uncomment the next two lines if you want to insert solution...
%\vskip 1cm
%{\bf Solution:}

% insert your solution here!


%--------------------------------------------------------------------------
\vskip 1cm
\hrule
{\bf Problem 4.}



Suppose we discretize the {\em nonlinear} pendulum problem $u''(x) =
-\gamma^2 \sin(u(x))$ (again with Dirichlet boundary conditions) using
the standard second order centered approximation to $u''(x_i)$, as 
discussed in Section 2.16. 

The Jacobian matrix $G'(U)$ now has the form $G'(U)=T+D$,
where $T$ is the standard tridiagonal matrix of (2.10) and $D$ is now a
diagonal matrix that is no longer a scalar multiple of the identity matrix.
It is similar to (2.82) but with an additional factor $\gamma^2$.

We can no longer compute the eigenvalues or eigenvectors of $A=T+D$ exactly.

Consider a problem for which $\gamma < \pi$ and suppose we also
know there is a unique exact solution $u(x)$ to the boundary value
problem being considered (there is in this case, but you don't need
to prove it).

Then, using the results of Problem 3, show that as $h \goto 0$ we
can find a uniform bound on $\|(G'(U))^{-1}\|_2$ and hence this
method for the nonlinear pendulum is stable in the sense of Definition 2.2.


% uncomment the next two lines if you want to insert solution...
%\vskip 1cm
%{\bf Solution:}

% insert your solution here!



%--------------------------------------------------------------------------

\end{document}
