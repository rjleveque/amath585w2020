\documentclass[10pt]{article}

\usepackage{graphicx}
\usepackage{amsmath,amsfonts,amssymb}

\usepackage{hyperref}  % for urls and hyperlinks


\setlength{\textwidth}{6.2in}
\setlength{\oddsidemargin}{0.3in}
\setlength{\evensidemargin}{0in}
\setlength{\textheight}{8.9in}
\setlength{\voffset}{-1in}
\setlength{\headsep}{26pt}
\setlength{\parindent}{0pt}
\setlength{\parskip}{5pt}




% a few handy macros

\newcommand\matlab{{\sc matlab}}
\newcommand{\goto}{\rightarrow}
\newcommand{\bigo}{{\mathcal O}}
\newcommand{\half}{\frac{1}{2}}
%\newcommand\implies{\quad\Longrightarrow\quad}
\newcommand\reals{{{\rm l} \kern -.15em {\rm R} }}
\newcommand\complex{{\raisebox{.043ex}{\rule{0.07em}{1.56ex}} \hskip -.35em {\rm C}}}


% macros for matrices/vectors:

% matrix environment for vectors or matrices where elements are centered
\newenvironment{mat}{\left[\begin{array}{ccccccccccccccc}}{\end{array}\right]}
\newcommand\bcm{\begin{mat}}
\newcommand\ecm{\end{mat}}

% matrix environment for vectors or matrices where elements are right justifvied
\newenvironment{rmat}{\left[\begin{array}{rrrrrrrrrrrrr}}{\end{array}\right]}
\newcommand\brm{\begin{rmat}}
\newcommand\erm{\end{rmat}}

% for left brace and a set of choices
\newenvironment{choices}{\left\{ \begin{array}{ll}}{\end{array}\right.}
\newcommand\when{&\text{if~}}
\newcommand\otherwise{&\text{otherwise}}
% sample usage:
%  \delta_{ij} = \begin{choices} 1 \when i=j, \\ 0 \otherwise \end{choices}


% for labeling and referencing equations:
\newcommand{\eql}{\begin{equation}\label}
\newcommand{\eqn}[1]{(\ref{#1})}
% can then do
%  \eql{eqnlabel}
%  ...
%  \end{equation}
% and refer to it as equation \eqn{eqnlabel}.  


% some useful macros for finite difference methods:
\newcommand\unp{U^{n+1}}
\newcommand\unm{U^{n-1}}

% for chemical reactions:
\newcommand{\react}[1]{\stackrel{K_{#1}}{\rightarrow}}
\newcommand{\reactb}[2]{\stackrel{K_{#1}}{~\stackrel{\rightleftharpoons}
   {\scriptstyle K_{#2}}}~}

% Parts:

% set enumerate to give parts a, b, c, ...  rather than numbers 1, 2, 3...
\renewcommand{\theenumi}{\alph{enumi}}
\renewcommand{\labelenumi}{(\theenumi)}

% set second level enumerate to give parts i, ii, iii, iv, etc.
\renewcommand{\theenumii}{\roman{enumii}}
\renewcommand{\labelenumii}{(\theenumii)}

  % input some useful macros

\begin{document}

% header:
\hfill\vbox{\hbox{AMath 585}
\hbox{Homework \#5}\hbox{Due Thursday, March 5, 2020}}

\vskip 5pt

Homework is due to Canvas by 11:00pm PDT on the due date.

To submit, see
\url{https://canvas.uw.edu/courses/1352870/assignments/5268734}


%--------------------------------------------------------------------------
\vskip 1cm
\hrule
{\bf Problem 1.}

Suppose $A \in \reals^{m\times m}$ is strictly row diagonally dominant,
i.e.,
\begin{equation}\label{diagdom}
\sum_{j\neq i} |a_{ij}| < |a_{ii}|, \quad\text{for}~i=1,~2,~\ldots,~m.
\end{equation} 
where the sum goes from $j=1$ to $m$ omitting the diagonal term.

Let $G$ be the Jacobi iteration matrix for this matrix $A$.

(a) Show that $\|G\|_\infty < 1$ and hence the Jacobi iteration converges,
since $\|e_k\|_\infty \leq \|G\|_\infty^k \|e_0\|_\infty$.  (Here and below
subscript $k$ refers to the $k$th iteration).

(b) Use the Gershgorin Theorem of Appendix C.8 to show that we also have
$\rho(G) < 1$ in the case when \eqn{diagdom} holds.

(c) Suppose $A$ is a normal matrix ($A^TA = AA^T$) and suppose something
stronger than \eqn{diagdom} holds, namely,
\begin{equation}\label{diagdom2}
\sum_{j\neq i} |a_{ij}| \leq \beta |a_{ii}|, \quad\text{for}~i=1,~2,~\ldots,~m
\end{equation} 
for $\beta = 1/2$.
Show that in this case Jacobi would reduce the 2-norm of the error
$\|e_k\|_2$ by a factor of at least $10^6$ in only 20 iterations, i.e., 
$\|e_k\|_2 \leq 10^{-6}\|e_0\|_2.$


(d) More generally, suppose the factor $\beta$ in \eqn{diagdom2}
is some value satisfying $\beta < 1$, and suppose the matrix is not
necessarily normal but we have an upper bound $\tilde \kappa$ on
the 2-norm condition number of $R$, the matrix of right eigenvectors
of $G$.  In terms of $\beta$ and $\tilde \kappa$, what is the maximum
number of iterations that would be required to reduce the 2-norm
error by a factor of $10^6$?  (In exact arithmetic.)

% uncomment the next two lines if you want to insert solution...
%\vskip 1cm
%{\bf Solution:}

% insert your solution here!


%--------------------------------------------------------------------------
\vskip 1cm
\hrule
{\bf Problem 2.}

Based on Problem 1, you might think that the more diagonally dominant a
matrix is, the better in terms of convergence rate.
However, consider these two matrices:
\[
A_1 = \bcm 1&-1/2\\ -1/2&1\ecm, \qquad
A_2 = \bcm 1&-3/4\\ -1/12&1\ecm.
\]
Determine the values $\beta$ for each case (as in \eqn{diagdom2},
and also the asymptotic convergence rates $\rho(G_1)$ and $\rho(G_2)$
in each case, where $G_i$ is the Jacobi iteration matrix for $A_i$.

% uncomment the next two lines if you want to insert solution...
%\vskip 1cm
%{\bf Solution:}

% insert your solution here!




%--------------------------------------------------------------------------
\vskip 1cm
\hrule
{\bf Problem 3.}  Suppose $A$ is a {\em singular} matrix and we do a splitting of
the form $A = M-N$ in such a way that $M$ is nonsingular (e.g. Jacobi
iteration in a case where the diagonal elements of $A$ are all nonzero).

Show that in spite of $M$ being nonsingular, the iteration matrix $G=M^{-1}N$
can never satisfy $\rho(G) < 1$.

Thus we cannot expect an iterative method for such a system to converge in
general, which makes sense if the matrix is singular.  However, see the next
problem...

% uncomment the next two lines if you want to insert solution...
%\vskip 1cm
%{\bf Solution:}

% insert your solution here!




%--------------------------------------------------------------------------
\vskip 1cm
\hrule
{\bf Problem 4.}
Suppose we want to solve the boundary value problem $u''(x) = f(x)$ on
$0\leq x \leq 1$ with {\em periodic boundary conditions}: $u(0) = u(1)$.
Recall that this has no solution unless a certain condition is satisfied by
$f(x)$, in which case it has infinitely many solutions.

Recall also that if we discretize this with the standard centered second-order
approximation, using a uniform grid with $h = 1/(m+1)$,
we get a tridiagonal matrix with additional corner terms from
the periodic boundary conditions. The matrix is singular and so the discrete
problem has an analogous solvability condition.

(a) Suppose we use Jacobi iteration to solve this problem, in a case where
the discrete solvability condition is satisfied.  What is the
iteration matrix $G$ for this problem? 

(b) Determine the eigenvalues and eigenvectors of $G$.
Appendix C.7 of the text might be useful.

(c) You should find that one eigenvalue of $G$ is equal to 1, and hence the
Jacobi iteration does not appear to converge according to the theory of
Section 4.2 (and consistent with Problem 3).  
But if the solvability condition is satisfied then in practice 
the method does converge to one of the infinitely many solutions of the
linear system.  Explain in what sense this is true and how the particular
solution obtained is related to the initial guess $u^{[0]}$ used for the
Jacobi iteration.
{\bf Hint:} Express the initial error relative to some particular solution
as a linear combination of the eigenvectors and then observe the effect of
iterating with the iteration matrix $G$.

(d) In the case it does converge, as described in (c), what is the
expected convergence rate?  (Note that since $\rho(G)=1$, this is not it.)



% uncomment the next two lines if you want to insert solution...
%\vskip 1cm
%{\bf Solution:}

% insert your solution here!


%--------------------------------------------------------------------------
\vskip 1cm
\hrule
{\bf Problem 5.}

Consider the problem
\[
\frac{d}{dx}\left(\kappa(x) u'(x)\right) = f(x)
\]
on $0\leq x \leq 1$ with $\kappa(x) >0$ everywhere. 

(a) Determine the solution of this problem for the case of $f(x)=0$ and
a discontinuous $\kappa(x)$ given by

\[
\kappa(x) = \begin{choices}
0.1 & \text{if~}  x < 0.5,\\
1& \text{if~}  x > 0.5.
\end{choices}
\]
with boundary conditions $u(0) = 0$ and $u(1) = 1$. The solution should be
continuous and piecewise linear with a discontinuity in slope at $x=0.5$
(Note this could model steady steady heat flow through a material that
conducts heat better on the right half than on the left.)


(b) Now suppose we discretize using the symmetric matrix $A$ of
(2.73) in the text.  Choose $m=19$ (odd, so there is a grid point exactly
at $x=0.5$, in which case the solution to the discrete system should
agree with the exact solution of the differential equation).

Write a computer code to solve this system using the SOR method for an
arbitrary value of $\omega$.  Note that setting $\omega = 1$ should just
reduce to the Gauss-Seidel method for comparison.  

Test this code and estimate the rate of convergence with $\omega = 1$ and
with $\omega = 0.7$, by doing a least squares fit of the convergence
history as demonstrated in the notebook {\tt IterativeMethods.ipynb}.


% uncomment the next two lines if you want to insert solution...
%\vskip 1cm
%{\bf Solution:}

% insert your solution here!


%--------------------------------------------------------------------------

\end{document}
